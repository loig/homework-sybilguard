\documentclass[a4paper,11pt]{article}
\usepackage[utf8]{inputenc}
\usepackage[T1]{fontenc}
\usepackage[english]{babel}
\usepackage{amsfonts}
\usepackage{graphicx}
\usepackage{ae}
\usepackage{fancyheadings}

\newcommand{\intf}{\raisebox{0.2mm}{\hspace{-5mm}$\rhd$\hspace{2mm}}}

\pagestyle{fancy}

\lhead{{\sc Dagand} \& {\sc Jezequel}}
\chead{PAP}
\rhead{Report: SybilGuard}

\begin{document}

\begin{center} 
  {\Large PAP \\ Report SybilGuard}\\
  Pierre-Evariste {\sc Dagand} \\
  Loïg {\sc Jezequel}
\end{center}

\section*{Introduction}

%% Sybil Attack
%%   - State its definition
%%   - Practical interest of canceling the Sybil Attacks

In peer-to-peer systems (and more generally in large distributed systems) a \emph{sybil attack} is a situation where a malicious user obtains multiple identities associated with multiples different nodes of the system.
Most byzantine failure tolerant protocols assume that no more than $1/3$ of the nodes are byzantine.
So, if a sybil attackant can control a large part of the system (more than $1/3$ of the nodes), he is able to break all defenses against byzantine failures.
Hence, providing efficient defenses against sybil attacks will improve possibilities of collaborative tasks based on voting schemes on peer-to-peer systems.

%% Impossibility Result
%%   - State the result: Douceur [IPTPS'02]
%%   - Consequence: to avoid Sybil Attacks, we must use a Central authority
%%   - Consequence: OR we should solve a weaker problem

In~\cite{douceur} an important result is prooved: without a central (and trusted) authority responsible for checking that a one-to-one correspondance between users and identities exists, one can always have multiple identities.
That is why \emph{SybilGuard}, which is a fully decentralized solution, proposes to solve a weaker problem.

%% SybilGuard trade-off
%%   - Solve a weaker problem: (very briefly) describe the problem solved
%%   - Still, useful: attenuate its supposed weaknesses
%%   - Demonstrate some uses: give example of what can be done with the guarantees it provides

Instead of totally avoid sybil identities SybilGuard suggests to bound the number of sybil groups (i.e. groups, defined by an equivalence relation, which contains sybil identities) and the size of these groups.
In fact even if SybilGuard does not fully prevent against sybil attacks it is sufficiently strong to be useful.
For example, in order to maintain replicas of a file, ensure that at most $g$ sybil groups exist allows to provide only $g+1$ replicas (to ensure that at most one is not on a sybil node).
Moreover having a bounded number of sybil nodes, smaller than the total number of nodes, ensures that, by randomly assign the replicas, the probability of having a majority of replicas on honest nodes will grow exponentially with the number of replicas.

%% Our plan
%%   - Present our plan
%%   - Particularity: the experimental results are mentioned along the text, 
%%                    to highlight the practical value of this system

\section{Design Rationale}

%% Social Network
%%   - based on human, strong trust relationship
%%   - particular quotient cut of ``normal'' social networks
%%   - definition of ``attack edges''
%%   - ``fast mixing'' property of social networks 
%%      - definition 
%%      - implication for the theoretical study

%% Random route
%%   - Principle (to compare with random walk)
%%   - ``Good'' length of a random walk
%%       - What is ``Good''
%%       - Theoretical results, at least in the case of a Social Network
%%   - Loops
%%       - Where do they appear
%%       - What is there impact (reduce efficiency, not soundness)
%%       - Experimental result: Section 6.1 paragraph 1

%% Guarantees provided by SybilGuard
%%   - Bound the number of Sybil groups
%%   - Bound the size of Sybil groups  
%%       - Experimental result: Section 6.2 paragraph 1
%%   - Accept honest nodes with hight probability
%%       - Experimental result: Section 6.1 paragraph 2, Section 6.2 paragraph 2

\section{Distributed Algorithm}

%% Protocol Overview

%% Data-Structures
%%   - Registry table
%%   - Witness table
%%   - Scalability considerations
%%      - Number of entries
%%      - Memory footprint

%% Determining the length of random routes
%%   - Principle
%%   - Experimental result: Section 6.1 paragraph 3, Section 6.2 paragraph 3


\section{Dealing with Dynamic Networks}

%% Routing tables maintenance
%%    - Algorithms for adding/deleting an edge
%%    - Generalization to adding/deleting a node
%%    - Can take days to complete, there is no hurry: work at the social level

%% Dealing with Offline nodes
%%    - State the cases where a communication is required
%%    - For each case, give the insight that a majority voting on a redundant random routing solve the problem
%%    - Describe the redundancy routing principle: Section 4.4
%%      - Mention that it also avoids attacks exploiting dynamism
%%      - Mention that it also helps nodes close to a Sybil zone

\section{Conclusion}

%% Scalable solution to fight Sybil attacks

%% Based on theoretical results

%% Lack of real-life deployment


\bibliography{report}{}

\bibliographystyle{plain}

\end{document}
