\documentclass[a4paper,11pt]{article}
\usepackage[utf8]{inputenc}
\usepackage[T1]{fontenc}
\usepackage[english]{babel}
\usepackage{amsfonts}
\usepackage{graphicx}
\usepackage{ae}
\usepackage{fancyheadings}

\newcommand{\intf}{\raisebox{0.2mm}{\hspace{-5mm}$\rhd$\hspace{2mm}}}

\pagestyle{fancy}

\lhead{{\sc Dagand} \& {\sc Jezequel}}
\chead{PAP}
\rhead{Report: SybilGuard}

\begin{document}

\begin{center} 
  {\Large PAP \\ Report SybilGuard}\\
  Pierre-Evariste {\sc Dagand} \\
  Loïg {\sc Jezequel}
\end{center}

\section*{Introduction}

%% Sybil Attack
%%   - State its definition
%%   - Practical interest of canceling the Sybil Attacks

In peer-to-peer systems (and more generally in large distributed systems) a \emph{sybil attack} is a situation where a malicious user obtains multiple identities associated with multiples different nodes of the system.
Most byzantine failure tolerant protocols assume that no more than $1/3$ of the nodes are byzantine.
So, if a sybil attackant can control a large part of the system (more than $1/3$ of the nodes), he is able to break all defenses against byzantine failures.
Hence, providing efficient defenses against sybil attacks will improve possibilities of collaborative tasks based on voting schemes on peer-to-peer systems.

%% Impossibility Result
%%   - State the result: Douceur [IPTPS'02]
%%   - Consequence: to avoid Sybil Attacks, we must use a Central authority
%%   - Consequence: OR we should solve a weaker problem

In~\cite{douceur} an important result is prooved: without a central (and trusted) authority responsible for checking that a one-to-one correspondance between users and identities exists, one can always have multiple identities.
That is why \emph{SybilGuard}, which is a fully decentralized solution, proposes to solve a weaker problem.

%% SybilGuard trade-off
%%   - Solve a weaker problem: (very briefly) describe the problem solved
%%   - Still, useful: attenuate its supposed weaknesses
%%   - Demonstrate some uses: give example of what can be done with the guarantees it provides

Instead of totally avoid sybil identities SybilGuard suggests to bound the number of sybil groups (i.e. groups, defined by an equivalence relation, which contains sybil identities) and the size of these groups.
In fact even if SybilGuard does not fully prevent against sybil attacks it is sufficiently strong to be useful.
For example, in order to maintain replicas of a file, ensure that at most $g$ sybil groups exist allows to provide only $g+1$ replicas (to ensure that at most one is not on a sybil node).
Moreover having a bounded number of sybil nodes, smaller than the total number of nodes, ensures that, by randomly assign the replicas, the probability of having a majority of replicas on honest nodes will grow exponentially with the number of replicas.

%% Our plan
%%   - Present our plan
%%   - Particularity: the experimental results are mentioned along the text, 
%%                    to highlight the practical value of this system

network (1): one million nodes, average node degree 24,

network (2): 10000 nodes, average node degree 24, 

network (3): 100 nodes, average node degree 12.

\section{Design Rationale}

%% Social Network
%%   - based on human, strong trust relationship
%%   - particular quotient cut of ``normal'' social networks
%%   - definition of ``attack edges''
%%   - ``fast mixing'' property of social networks 
%%      - definition 
%%      - implication for the theoretical study

SybilGuard is based on a social network, that is a graph where the topology is constructed from human-established trust relations: there is an edge between two entities if and only if they know eachother (e.g. they have friendship relations in real life).
The principe of SybilGuard is to let a node $V$ accept a node $S$ if and only if a random walk from $V$ in the social network intersects with a random walk from $S$.

Generally social networks have  big \emph{quotient cut}: a large number of nodes can not be disconnected from the rest of the graph by removing a small number of edges.
In presence of many sybil nodes the quotient cut of such a network becomes small: a small set of edges (called \emph{attack edges}) disconnect a large part of the graph (the sybil nodes) from the rest of it when removed.

Moreover, social networks ensure a property called \emph{fast mixing}: after a small number of hops on a random walk ($O(\log n)$ hops) the walk becomes roughly independent from its starting point.
In particular it means that in a social network two random walks starting from two different nodes have high probability to instersect eachother after a small number of hops.

%% Random route
%%   - Principle (to compare with random walk)
%%   - ``Good'' length of a random walk
%%       - What is ``Good''
%%       - Theoretical results, at least in the case of a Social Network
%%   - Loops
%%       - Where do they appear
%%       - What is there impact (reduce efficiency, not soundness)
%%       - Experimental result: Section 6.1 paragraph 1

\paragraph{}
In order to be able to exploit the two properties defined above, SybilGuard uses the concept of \emph{random routes}.
Random routes are walk in the graph based on random routing tables: at the begining each node computes his routing table (which is a random permutation of his edges) and then these routing tables are used to computes the random routes.
In fact if a node initiates a walk $r$ by a given edge with a fixed length then all walk initiaded by the same node with the same edge and the same length will be exactly $r$.

Because SybilGuard is based on intersection between random routes it is necessary to define the optimal length of these walks: not to short (a random routes has to intersect with a sufficient number of other random routes to ensure each node to accept a sufficient number of nodes) and not to long (if a random route goes into the sybil part of the network then a node will accept a huge number of sybil nodes).
Some analytical results show that a length of $O(\sqrt{n} \log n)$ for random walks ensure the above requirements. 
These results are for random walks but can be extended to random routes.

Random routes can also have loops.
The first thing we can notice is that a loop can only appar at the start point of a random route (because of the properties of random routes).
Moreover such a loop involves at least three hops.
One can also notice that if $d$ is the smallest degree in the graph a three hops loop is formed with probability at most $1/d^2$ and a bigger loop has less probability to be formed.
The problem with loops is that they reduce the probability for a random route to intersect with many other random routes. 
Experimental results, maid without sybil attackers, seems to confirm that there is really few random routes which are loops: in network (1) $99,3\%$ of the 2500 hops routes does not loop, in network (2) $99,7\%$ of the 200 hops routes does not loop, and in network (3) $90\%$ of the 50 hops routes does not loop.


%% Guarantees provided by SybilGuard
%%   - Bound the number of Sybil groups
%%   - Bound the size of Sybil groups  
%%       - Experimental result: Section 6.2 paragraph 1
%%   - Accept honest nodes with hight probability
%%       - Experimental result: Section 6.1 paragraph 2, Section 6.2 paragraph 2

\paragraph{}
Using the concept of random routes and the properties of social networks, SybilGuard provides some properties of practical interest.

SybilGuard allows the number of sybil groups to be bounded by the number of attack edges.
This is ensured by the properties of random routes: if two routes share an edge then they share all there edges after it.
Because there is a bounded number $g$ of attack edges, the random routes starting at sybil nodes can only follow $g$ different paths in the honest part of the social network.
Thus, SybilGuard will let node $V$ put in the same group all the nodes which random route intersect the random route of $V$ at the same node and coming from the same edge.
Finally, for $V$, there will be at most $g$ groups of accepted nodes which can contain sybil nodes. 

Moreover SybilGuard bounds the size of sybil groups.
Denote by $w$ the length of the random routes.
In fact, given a random route $r$, we deduce from the properties of random routes that there can be at most $w$ distinct routes that intersect with $r$ at the same node, entering this node from the same edge.
Hence the owner of $r$ will accept only $w$ nodes in each group: it will bound the number of sybil nodes in a sybil group (by $w$) and will not make a node refuse honest nodes in an honest group.
Experiments give the following results: in network (1), with 2500 attack edges $0,2\%$ of the nodes can accept more than $g.w$ sybil nodes and with 2000 attack edges it is $0\%$.
In network (2) with 204 attack edges $0,4\%$ of the nodes are not protected, and in network (3) with 11 attack edges $5,1\%$ of the nodes are not protected.

SybilGuard also garantees that, using appropriate length for random routes, an honest node will accept many of the honest nodes with high probability. 
This was verified experimentaly.
Without sybil attackers, in network (1) a node has $99,96\%$ probability to be accepted by the node using SybilGuard with a length of random routes of 300 hops, in network (2) it is $99,29\%$ with a length of 30 hops, and in network (3) it is $99,97\%$ with a length of 15 hops.
When there is sybil attackers, the probability for an honest node to be accepted by the node using SybilGuard is, in network (1) with 2500 attack edges, $99,8\%$.
In the network (2) it is $99,6\%$ with 204 attack edges, and in network (3) it is $87,7\%$ with 11 attack edges.
One can notice that it doesn't mean that some honest nodes will not be accepted by the system but that the node using SybilGuard will not accept these nodes.

\section{Distributed Algorithm}

%% Protocol Overview

%% Data-Structures
%%   - Registry table
%%   - Witness table
%%   - Scalability considerations
%%      - Number of entries
%%      - Memory footprint

%% Determining the length of random routes
%%   - Principle
%%   - Experimental result: Section 6.1 paragraph 3, Section 6.2 paragraph 3


\section{Dealing with Dynamic Networks}

%% Routing tables maintenance
%%    - Algorithms for adding/deleting an edge
%%    - Generalization to adding/deleting a node
%%    - Can take days to complete, there is no hurry: work at the social level

%% Dealing with Offline nodes
%%    - State the cases where a communication is required
%%    - For each case, give the insight that a majority voting on a redundant random routing solve the problem
%%    - Describe the redundancy routing principle: Section 4.4
%%      - Mention that it also avoids attacks exploiting dynamism
%%      - Mention that it also helps nodes close to a Sybil zone

\section{Conclusion}

%% Scalable solution to fight Sybil attacks

%% Based on theoretical results

%% Lack of real-life deployment


\bibliography{report}{}

\bibliographystyle{plain}

\end{document}
