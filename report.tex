\documentclass[a4paper,11pt]{article}
\usepackage[utf8]{inputenc}
\usepackage[T1]{fontenc}
\usepackage[english]{babel}
\usepackage{amsfonts}
\usepackage{graphicx}
\usepackage{ae}
\usepackage{fancyheadings}

\newcommand{\intf}{\raisebox{0.2mm}{\hspace{-5mm}$\rhd$\hspace{2mm}}}

\pagestyle{fancy}

\lhead{{\sc Dagand} \& {\sc Jezequel}}
\chead{PAP}
\rhead{Report: SybilGuard}

\begin{document}

\begin{center}
  {\Large Report: SybilGuard}
\end{center}
\begin{center}
  {\large Pierre-Evariste {\sc Dagand} \qquad Loïg {\sc Jezequel}}
\end{center}


\section*{Introduction}

%% Sybil Attack
%%   - State its definition
%%   - Practical interest of canceling the Sybil Attacks

In peer-to-peer systems -- and more generally in distributed systems
-- a \emph{Sybil attack} is a situation where a malicious user obtains
multiple identities associated with distinct nodes of the system.
However, for example, most Byzantine failure tolerant protocols assume
that no more than $1/3$ of the nodes are Byzantine.  So, if a Sybil
attacker is able to control more a third of the system, he could
defeat the defenses against Byzantine failures.  Hence, providing an
efficient mechanism against Sybil attacks paves the way for
collaborative tasks based on voting schemes in peer-to-peer systems.

%% Impossibility Result
%%   - State the result: Douceur [IPTPS'02]
%%   - Consequence: to avoid Sybil Attacks, we must use a Central authority
%%   - Consequence: OR we should solve a weaker problem

In~\cite{douceur} an important result is proved: without a central,
trusted authority responsible of ensuring a one-to-one correspondence
between users and identities, it is \emph{impossible} to prevent Sybil
attacks. Therefore, instead of relying on a central authority,
\emph{SybilGuard} proposes to weaken the problem in order to allow
fully decentralized solutions, while still providing a useful service.

%% SybilGuard trade-off
%%   - Solve a weaker problem: (very briefly) describe the problem solved
%%   - Still, useful: attenuate its supposed weaknesses
%%   - Demonstrate some uses: give example of what can be done with the guarantees it provides

Hence, SybilGuard offers to bound the number of Sybil groups
(i.e. groups which contains the Sybil identities of the same malicious
user) and the size of these groups.  Hence, even if this does not
fully protect against Sybil attacks, it is sufficient to be useful.
For example, in order to maintain replicas of a file, knowing that at
most $g$ Sybil groups exist allow us to distribute only $g+1$ replicas
in distinct groups.  Moreover, having a bounded number of Sybil nodes
guarantees that, by randomly assigning the replicas, the probability of
having a majority of replicas on honest nodes grows exponentially with
the number of replicas.

%% Our plan
%%   - Present our plan
%%   - Particularity: the experimental results are mentioned along the text, 
%%                    to highlight the practical value of this system

In the following, we first present the design rationale of SybilGuard.
Then, we describe its decentralized implementation. Finally, we
explain how it is possible to deal with dynamic networks. 


\section{Design Rationale}

%% Social Network
%%   - based on human, strong trust relationship
%%   - particular quotient cut of ``normal'' social networks
%%   - definition of ``attack edges''
%%   - ``fast mixing'' property of social networks 
%%      - definition 
%%      - implication for the theoretical study

SybilGuard is based on a social network: a graph of
human-established trust relationships (an edge exists between two
entities if and only if they trust each other \emph{in real
life}). Usually, social networks have big \emph{quotient cut}: a large
number of nodes can not be disconnected from the graph by
removing a small number of edges. However, the quotient cut between
honest and Sybil nodes is inherently small: a small set of edges --
termed \emph{attack edges} -- connect the Sybil nodes to the honest
nodes, as we expect the number of trust relationships between honest
persons and fake identities to be very low. Moreover, social networks
are \emph{fast mixing}: after a small number of random hops ($O(\log
n)$), the resulting position is independent from the starting
point.  In particular, this means that two random walks starting from
two distinct nodes intersect with high probability.

%% Random route
%%   - Principle (to compare with random walk)
%%   - ``Good'' length of a random walk
%%       - What is ``Good''
%%       - Theoretical results, at least in the case of a Social Network
%%   - Loops
%%       - Where do they appear
%%       - What is there impact (reduce efficiency, not soundness)
%%       - Experimental result: Section 6.1 paragraph 1

In order to exploit these properties, SybilGuard uses \emph{random
routes}. A random route consists of a walk in the social graph,
following \emph{random routing tables}. For a given node, a random routing
table is defined as a random permutation $\sigma$ of its social edges,
meaning that a walk coming from edge $e_1$ will be redirected to edge
$\sigma(e_1)$. Hence, unlike a random walk, a random route is fully
deterministic once the routing tables are built.
As SybilGuard exploits the intersection property, the length of these
routes has to be carefully chosen: not to short -- random routes have
to intersect with high probability -- and not to long -- going into
the Sybil region would lower its efficiency.  The paper shows that a
length of $O(\sqrt{n} \log n)$ hops ensures the above requirements.

%% Loops: not so important after all

%% Random routes can also have loops.  The first thing we can notice is
%% that a loop can only appar at the start point of a random route
%% (because of the properties of random routes).  Moreover such a loop
%% involves at least three hops.  One can also notice that if $d$ is the
%% smallest degree in the graph a three hops loop is formed with
%% probability at most $1/d^2$ and a bigger loop has less probability to
%% be formed.  The problem with loops is that they reduce the probability
%% for a random route to intersect with many other random routes.
%% Experimental results, maid without Sybil attackers, seems to confirm
%% that there is really few random routes which are loops: in network (1)
%% $99,3\%$ of the 2500 hops routes does not loop, in network (2)
%% $99,7\%$ of the 200 hops routes does not loop, and in network (3)
%% $90\%$ of the 50 hops routes does not loop.


%% Guarantees provided by SybilGuard
%%   - Bound the number of Sybil groups
%%   - Bound the size of Sybil groups  
%%       - Experimental result: Section 6.2 paragraph 1
%%   - Accept honest nodes with hight probability
%%       - Experimental result: Section 6.1 paragraph 2, Section 6.2 paragraph 2

Based on random routes and the properties of social networks,
SybilGuard provides some properties of practical interest. First,
SybilGuard bounds the number of Sybil groups by the number of attack
edges.  This is ensured by the \emph{convergence property} of random
routes: if two routes go through a single edge, then they share all
subsequent edges.  Because there is a bounded number $g$ of attack
edges, the random routes starting at Sybil nodes can only follow $g$
distinct paths in the honest region.  Thus, for a verifier $V$, all
nodes intersecting at the same node $X$ belongs to the same group and,
among all these groups, at most $g$ contains Sybil nodes. Some
experimental results are discussed in Section 6.2, paragraph 1. They
clearly demonstrates SybilGuard's efficiency in bounding the number of
Sybil groups.

Moreover, SybilGuard bounds the size of each Sybil group. Let us
denote $w$ the length of random routes.  Given an edge $e$, we deduce
from the \emph{back-traceability property} of random routes -- going
through an edge fully determine which edge was previously traversed --
that there is at most $w$ distinct routes going through $e$.  Hence, a
verified will accept at most $w$ nodes by group. This has been
experimentally verified in Section 6.1 paragraph 2 (with only honest
nodes) as well as in Section 6.2 paragraph 2 (with malicious
users). In both case, this gave excellent results.

SybilGuard also guarantees that an honest node will accept other honest
nodes with high probability: it is sufficient for their routes to
intersect.

%% Experimentally, 
%% Without Sybil attackers, in network (1) a node has $99,96\%$ probability to be accepted by the node using SybilGuard with a length of random routes of 300 hops, in network (2) it is $99,29\%$ with a length of 30 hops, and in network (3) it is $99,97\%$ with a length of 15 hops.
%% When there is Sybil attackers, the probability for an honest node to be accepted by the node using SybilGuard is, in network (1) with 2500 attack edges, $99,8\%$.
%% In the network (2) it is $99,6\%$ with 204 attack edges, and in network (3) it is $87,7\%$ with 11 attack edges.
%% One can notice that it doesn't mean that some honest nodes will not be accepted by the system but that the node using SybilGuard will not accept these nodes.

\section{Distributed Algorithm}

%% Protocol Overview

The idea of SybilGuard protocol is to perform random routes in a fully
decentralized way and to verify their intersection using two
data-structures, the \emph{registry table} and the \emph{witness
table}. The verification process is the following: a verifier $V$ of
degree $d$ performs $d$ random routes of length $w$. It accepts a
suspect $S$ if at least a half of the routes of $V$ intersect with
routes of $S$, after having ensured that the route announced by $S$ is
indeed feasible.

%% Data-Structures
%%   - Registry table
%%   - Witness table
%%   - Scalability considerations
%%      - Number of entries
%%      - Memory footprint

Each node $N$ maintains a registry table for each of its edge $e$.  The
$i^{th}$ entry of this table contains the identity of the node which
$i^{th}$ hop arrives to $N$ by $e$.  Each node also maintains a witness
table for each of its edges $e$.  The $i^{th}$ entry of this table
contains the identity of the node encountered at $i^{th}$ hop along
$e$.  These two data-structures allow $V$ to be sure that $S$ does not
falsify its routes. $V$ checks if $S$ witness tables -- which contain
the random routes of $S$ -- are consistent with the registry tables --
which contain the random routes arriving at the corresponding node --
of the intersection nodes.

These tables are simply built by propagation along neighbors.
Moreover, these tables need only to contain a public key per entry (and an IP address for witness tables). Hence,
maintaining these tables is not so expensive, in term of communication
as well as network usage. In a one million
nodes network, using random routes of length 2000, and 1024-bit
keys, the size of a registry table is roughly 256KB.  Hence, a 10 neighbors node will have to send 2.56MB of data.

%% Determining the length of random routes
%%   - Principle
%%   - Experimental result: Section 6.1 paragraph 3, Section 6.2 paragraph 3

SybilGuard is fully decentralized, thus each node $A$ has to locally
estimate the optimal length of its random routes.  To do so, $A$
performs a short \emph{random walk}, e.g. less than 3 hops, reaching
node $B$. Then, $A$ and $B$ both perform a random route in order to
determine how long these routes need to be to intersect.  After having
obtained multiple samples, $A$ computes their median $m$ and takes
$2.1 \times m$ as the length of its random routes. As shown in the
paper, the estimated length is optimal, with high probability. This
technique is experimentally confirmed in Section 6.1 paragraph 3 and
Section 6.2 paragraph 3: they show that the difference between the
estimation and the theoretical value is negligible.

%% Some experiments were done to check if this way of determining random
%% routes length was efficient.  The protocole is simple: first $B$ was
%% choosen uniformly random and a theorical length was computed, then $B$
%% was choosen using the method described above and the lenght obtained
%% was compared with the theorical one.  In the case where no Sybil
%% attackers exist, in network (1), using 30 samples the diffence with
%% the theorical length was at most 300 hops (for a theorical length of
%% 1906 hops) and using 100 samples the difference was at moste 150 hops.
%% In network (2), for a theorical length of 197 hops, the difference was
%% at most 30 hops, using 35 samples.  In network (3), for a theorical
%% length of 24 hops, the difference was at most 7 hops, using 40
%% samples.  Then in presence of Sybil attackers the probability for
%% sample to be potentially influenced by the adversary (i.e. any of the
%% two random routes enter in the Sybil region) was studied.  In network
%% (1), with 2500 attack edges, this probability is under $20\%$, as in
%% networks (2) and (3) with 204 and 11 attack edges.


\section{Dealing with Dynamic Networks}

In fact, SybilGuard as presented above does not deals with dynamism.
In this section we explain how it is done. First, it is important to
notice that, unless during the verification process, SybilGuard does not
rely on nodes being online or even their exact network
location. Hence, it needs only to track changes at the social level,
where we expect relations to be quite static.

%% Routing tables maintenance
%%    - Algorithms for adding/deleting an edge
%%    - Generalization to adding/deleting a node
%%    - Can take days to complete, there is no hurry: work at the social level

Hence, we can focus on the two following actions: adding or deleting a
trust edge. Adding an edge should be done carefully: a wrong choice
could require to modify a huge number of witness and registry tables.
Hence, the modification of $A$'s routing table is critical. The
solution proposed is the following: when edge $d+1$ is added to $A$ of
degree $d$, $A$ randomly choose an edge $k$ between 1 and $d+1$ and
replace the initial routing table $(x_1,x_2,\dots, x_d)$ by the new
routing table $(x_1,x_2,\dots x_{k-1},d+1,x_{k+1},\dots, x_d, x_k)$ if
$k\neq d+1$ or else by the new routing table $(x_1,x_2,\dots
,x_d,d+1)$.  When edge $d+1$ is deleted from $A$ of degree $d+1$ with
$x_k=d+1$, the routing table becomes $(x_1,x_2,\dots, x_d)$ if $k=d+1$
and else $(x_1,x_2,\dots, x_{k-1}, x_{d+1}, x_{k+1},\dots x_d)$. It is
shown in the technical report that these modifications conserve the
randomness of the routing table. Once we are able to add and delete
edges, we can similarly add and delete a node, by simply adding or
deleting its edges one by one.

%% Dealing with Offline nodes
%%    - State the cases where a communication is required
%%    - For each case, give the insight that a majority voting on a redundant random routing solve the problem
%%    - Describe the redundancy routing principle: Section 4.4
%%      - Mention that it also avoids attacks exploiting dynamism
%%      - Mention that it also helps nodes close to a Sybil zone

Concerning offline nodes, we first observe that a communication is
needed only in the following cases: when a node wants to verify a
suspected node, in which case it has to contact the intersection
nodes, and when a node propagates its tables to its neighbors, for
maintenance purpose.

The second case occurs when the social network is modified. However,
social relations in real life take months to change and an user
lifetime in a peer-to-peer system is about a year. Therefore, a
maintenance can take days to complete, without impacting the
performance or reliability of the system. Therefore, we do not have to
care about offline nodes: we simply wait for them to be online.

The first case is solved thanks to \emph{redundancy}: a verifier $V$
with degree $d$ performs $d$ random routes and accepts a node $S$ only
if they have more than $d/2$ intersections.  Hence, it is not a
problem if some intersections nodes are offline as soon as
sufficiently many are online. As side effect, this redundancy system
prevents against attacks exploiting dynamics and provides a
way for nodes close to a Sybil region to reduce their amount of
accepted Sybil nodes.

\section*{Conclusion}

%% Scalable solution to fight Sybil attacks

SybilGuard provides a fully decentralized mechanism to bound the
impact of Sybil attacks. Moreover, its protocol is lightweight and non
intrusive, both in term of network or memory usage. Hence, it can
scale perfectly to very large networks, without sacrificing either
performance or safety.

%% Based on theoretical results

It is important to notice that the bound provided by SybilGuard are
\emph{best effort}: they hold with high-probability. However, they are
strongly supported by a set of mathematical proofs, based on the
widely studied properties of social networks. Therefore, SybilGuard
offers a significant level of reliability, confirmed by several
experimental results: the number and size of Sybil groups are bounded
for $99,8\%$ of honest nodes while honest nodes are accepted by at
least $99,8\%$ of other honest nodes.

%% Lack of real-life deployment

Finally, we are looking forward real life deployment of such system,
in order to confirm -- or infirm -- its impressive capabilities.



\bibliography{report}{}
\bibliographystyle{abbrvnat}

\end{document}
